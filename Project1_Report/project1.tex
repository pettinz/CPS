%----------------------------------------------------------------------------------------
%	PACKAGES AND DOCUMENT CONFIGURATIONS
%----------------------------------------------------------------------------------------

\documentclass{article}

\usepackage[a4paper, total={6in, 8in}]{geometry}
\usepackage[italian]{babel}
\usepackage[version=3]{mhchem} % Package for chemical equation typesetting
\usepackage{natbib} % Required to change bibliography style to APA
\usepackage{amsmath} % Required for some math elements
\usepackage[separate-uncertainty=true]{siunitx}
\usepackage{tikz}
\usepackage{pgfplots}
\usepackage{graphicx}
% Here: H option for float placement
\usepackage{float}

% caption and subcaption work together
\usepackage{subcaption} % loads the caption package

\linespread{1.2}

\sisetup{output-decimal-marker = {,}}
\DeclareSIUnit\div{div}
\pgfplotsset{width=8cm,compat=1.9}
\setlength\parindent{0pt} % Removes all indentation from paragraphs
% \renewcommand{\labelenumi}{\alph{enumi}.} % Make numbering in the enumerate environment by letter rather than number (e.g. section 6)

%\usepackage{times} % Uncomment to use the Times New Roman font

%----------------------------------------------------------------------------------------
%	DOCUMENT INFORMATION
%----------------------------------------------------------------------------------------

\title{ Project I: distributed localization with CPSs \\ Modeling and control of cyberphysical systems \\ 01UDSOV} % Title

\author{%
\begin{tabular}{c} Simone Gallo \\ s276217 \end{tabular} \and
\begin{tabular}{c} Francesco Menon \\ s277870 \end{tabular} \and
\begin{tabular}{c} Angelo Pettinelli \\ s269291 \end{tabular} \and
\begin{tabular}{c} Esmeraldi Xuna \\ s277995 \end{tabular}}

\date{\today} % Date for the report

\begin{document}

\vspace*{\fill}
    \vbox{
        \centering
        \includegraphics[width=.75\textwidth]{logo}
        \maketitle %this typesets the contents of \title, \author and \date
    }
\vspace*{\fill}

\clearpage

% If you wish to include an abstract, uncomment the lines below
% \begin{abstract}
% Abstract text
% \end{abstract}

%----------------------------------------------------------------------------------------
%	SECTION 1
%----------------------------------------------------------------------------------------

\section{Introduction}
In this project, we simulate an indoor localization/tracking system through a wireless sensor network (WSN).
The sensor acquire the received signal strength (RSS) on a signal broadcast by a target to be located.

\subsection{Aims of the project}

\begin{enumerate}
    \item Simulation of a localization/tracking problem in WSNs
    \item Implementation of a localization/tracking distributed algorithm
    \item Analysis of the results
\end{enumerate}

\subsection{Physical setting}

\begin{itemize}
    \item Environment: square room of \SI{100}{\metre\squared}
    \item Grid: $p = 100$ square cells of \SI{1}{\metre\squared}
    \item Reference points: centers of the cells
    \item RSS model: indoor empirical model defined by the IEEE 802.15.4 standard
        \begin{equation}
            RSS(d) = \begin{cases}
                P_t - 40.2 - 20\log{d} + \eta & , \text{if $d\leq \SI{8}{\metre}$}\\
                P_t - 58.5 - 33\log{d} + \eta & , \text{if $d > \SI{8}{\metre}$}
            \end{cases}
        \end{equation}
        where $P_t=25$, $\eta$ is a Gaussian noise $\eta \sim \mathcal{N}(0,\sigma^2)$, $\sigma = 0.5$.
\end{itemize}

\subsection{WSN}

\begin{itemize}
    \item $n = 25$ sensors
    \item Deployment:
    \begin{itemize}
        \item uniformly at random positions; each sensor is connected with sensors at distance $\le r$.
        \item grid topology: the sensors are deployed on a grid $5 \times 5$; sensors are connected to 
        4 closest sensors (3 or 2 on the boundaries)
    \end{itemize}
\end{itemize}

%----------------------------------------------------------------------------------------

\end{document}
